\documentclass{article}
\usepackage{amsmath,amssymb,amsthm}
\usepackage{mathtools}
\usepackage{listings}
\usepackage{enumitem}

%--------------------------------------------------------------------
%  Mathematical specification of the \texttt{M31} field (GF(2^{31}-1))
%  Target language for implementation: Haskell
%--------------------------------------------------------------------

\begin{document}

\section{Algebraic setting}

\subsection{Prime modulus}
\[
  P \;=\; 2^{31}-1 \;=\; 2147483647
\]
$P$ is a Mersenne prime.  
Define the finite field
\[
  \mathbb{F}_{P} \;=\; \{\,0,1,\dots,P-1 \,\},
\]
with the usual addition and multiplication modulo~$P$.

\subsection{Canonical representation}
Every field element is represented by an unsigned
32-bit integer%
\footnote{Because $P<2^{32}$ the representation is injective.}
\[
   \operatorname{rep}\colon\mathbb{F}_{P}\longrightarrow\{0,\dots,2^{32}-1\},
   \qquad \operatorname{rep}(x)=x.
\]
The \emph{canonical range} is
$
  \mathcal{C}= [0,P-1].
$
Operations \textbf{must} return results in~$\mathcal{C}$.

\vspace{-0.25\baselineskip}
\section{Core operations}

Throughout, let $x,y\in\mathcal{C}$ and $z\in\{0,\dots,2P-1\}$.

\begin{enumerate}[label=\textbf{O\arabic*}.]
\item \textbf{Partial reduction}  
      \[
         \operatorname{partialReduce}(z)=
         \begin{cases}
             z-P &\text{if } z\ge P,\\[4pt]
             z   &\text{otherwise.}
         \end{cases}
      \]
      \textit{Pre-condition:} $z\in[0,2P)$.\medskip

\item \textbf{Full reduction}  
      For $w\in[0,P^{2})$,
      \[
        \operatorname{reduce}(w)
        \;=\;
        \Bigl(\bigl(\tfrac{w}{2^{31}}\bigr)+w+1\Bigr)
        \bmod 2^{31}
        \;\;=\;\;
        w \bmod P.
      \]
      (\emph{Explanation:} because $2^{31}\equiv1\pmod P$ we may
      fold every $2^{31}$-block into the low word.)\medskip

\item \textbf{Addition}  
      \[
        x\oplus y \;=\;
        \operatorname{partialReduce}(x+y).
      \]

\item \textbf{Negation}  
      \[
        \ominus x \;=\;
        \operatorname{partialReduce}(P-x).
      \]

\item \textbf{Subtraction}  
      \[
        x\ominus y
        \;=\;
        \operatorname{partialReduce}(x+P-y).
      \]

\item \textbf{Multiplication}  
      \[
        x\otimes y \;=\; \operatorname{reduce}(xy),
        \quad xy\text{ computed in }[0,P^{2}).
      \]

\item \textbf{Squaring} $\operatorname{square}(x)=x\otimes x$.

\item \textbf{Inverse} (Fermat)  
      \[
        \operatorname{inv}(x) = x^{P-2}\pmod P,
        \qquad x\neq0.
      \]
      An optimised addition-chain requiring $37$ multiplications is
      supplied below.\medskip

\item \textbf{Complex conjugation}  
      The field is prime\;⇒\;identity map.
\end{enumerate}

\subsection{Optimised exponent $P-2=2147483645$}

Let $\operatorname{sqn}^{(k)}(x)=x^{2^{k}}$.  
Define
\[
\begin{aligned}
t_{0}&=\operatorname{sqn}^{(2)}(x)\cdot x,\\
t_{1}&=\operatorname{sqn}^{(1)}(t_{0})\cdot t_{0},\\
t_{2}&=\operatorname{sqn}^{(3)}(t_{1})\cdot t_{0},\\
t_{3}&=\operatorname{sqn}^{(1)}(t_{2})\cdot t_{0},\\
t_{4}&=\operatorname{sqn}^{(8)}(t_{3})\cdot t_{3},\\
t_{5}&=\operatorname{sqn}^{(8)}(t_{4})\cdot t_{3},\\[4pt]
\operatorname{inv}(x)&=\operatorname{sqn}^{(7)}(t_{5})\cdot t_{2}.
\end{aligned}
\]
Correctness: each multiplication and square
raises $x$ to a power whose binary expansion equals
$P-2=1111111111111111111111111111101_{(2)}$.

\section{Auxiliary conversions}

\begin{description}[style=nextline,labelwidth=3.2cm,leftmargin=3.3cm]
\item[Int\,$\to$\,$\mathbb{F}_{P}$:]
      \(
        \operatorname{fromInt}(n)=
        \begin{cases}
            \operatorname{reduce}(\lvert n\rvert) & n\ge0,\\[4pt]
            \operatorname{reduce}(2P-\lvert n\rvert) & n<0.
        \end{cases}
      \)

\item[$\mathbb{F}_{P}$\,$\to$\,little-endian bytes:]
      store $\operatorname{rep}(x)$ in $4$~bytes.
\end{description}

\section{Complexities}

All arithmetic except inversion is $O(1)$.  
\(\operatorname{inv}\) costs \(37\) multiplies \(+\;31\) squares.

\section{Reference properties}

For all $a,b,c\in\mathbb{F}_{P}$:
\[
\begin{aligned}
&\text{P1 } a\oplus b = b\oplus a
&&\text{(comm.\,+)}\\
&\text{P2 } (a\oplus b)\oplus c = a\oplus(b\oplus c)
&&\text{(assoc.\,+)}\\
&\text{P3 } a\oplus0 = a,\; a\ominus a = 0
&&\text{(ident.\,+/inv.)}\\
&\text{P4 } a\otimes b = b\otimes a
&&\text{(comm.\,$\times$)}\\
&\text{P5 } (a\otimes b)\otimes c = a\otimes(b\otimes c)
&&\text{(assoc.\,$\times$)}\\
&\text{P6 } a\otimes1 = a,\; a\neq0\Rightarrow a\otimes\operatorname{inv}(a)=1
&&\text{(ident.\,$\times$/inv.)}\\
&\text{P7 } a\otimes(b\oplus c)=a\otimes b\oplus a\otimes c
&&\text{(distrib.)}
\end{aligned}
\]

\section{Testing specification (Haskell‐centric)}

\subsection{Property-based tests (QuickCheck)}

\begin{enumerate}[label=\textbf{Q\arabic*}.]
\item \textbf{Canonical range}  
      \(\forall x\;.\;0\le x < P\).
\item \textbf{Field laws}  
      Verify P1–P7 above for \texttt{100\,000} random triples.
\item \textbf{Inverse correctness}  
      \(\forall x\neq0.\;x\otimes\operatorname{inv}(x)=1\).
\item \textbf{Partial vs.\ full reduction}  
      Generate $z\in[0,2P)$ and check  
      \(\operatorname{partialReduce}(z)=z\bmod P\).
\item \textbf{Reduce identity}  
      Generate $w\in[0,P^{2})$ and check  
      \(\operatorname{reduce}(w)=w\bmod P\).
\item \textbf{Optimised exponent}  
      \(\forall x.\;\operatorname{inv}(x)=x^{P-2}\pmod P\).
\end{enumerate}

\subsection{Deterministic unit tests}

Mimic the Rust suite.

\begin{enumerate}[label=\textbf{T\arabic*}.]
\item \textbf{Edge cases}:  
      $0,1,P-1$ for every operation.
\item \textbf{Conversion tests}:  
      $\operatorname{fromInt}(-1)=P-1$,  
      $\operatorname{fromInt}(-10)=P-10$ etc.
\item \textbf{Byte serialisation}:  
      Round-trip $100$ random elements.
\item \textbf{Mul/Add/Neg consistency}:  
      For $10\,000$ pseudorandom pairs $(x,y)$ compare
      against \(\bmod\)-math using \verb|Integer|.
\end{enumerate}

\subsection{Performance assertions (optional)}

\begin{itemize}[nosep]
\item \(<\!20\)\,ns for addition/negation on current desktop~CPU.
\item \(<\!60\)\,ns for multiplication.
\item \(<\!1\)\,$\mu$s for inversion.
\end{itemize}

\section{Haskell implementation hints}

\begin{itemize}[nosep]
\item Use \verb|newtype M31 = M31 Word32| with \verb|UNPACK| pragma.
\item Mark arithmetic with \verb|{-# INLINE #-}|.
\item Implement \verb|Num|, \verb|Fractional|, \verb|Enum|, \verb|Random|.
\item Encode \(\operatorname{reduce}\) via the bit-trick:
      \begin{lstlisting}[language=Haskell,basicstyle=\ttfamily\small]
reduce :: Word64 -> Word32
reduce w = fromIntegral $
           ((w `shiftR` 31) + w + 1) `shiftR` 31 + w
           .&. 0x7fffffff
      \end{lstlisting}
\item For \verb|QuickCheck| generators choose \verb|choose (0,P-1)|.
\end{itemize}

\end{document}
